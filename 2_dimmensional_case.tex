\documentclass[12 pt, russian]{article}
\usepackage[russian]{babel}
\usepackage[utf8]{inputenc}
\usepackage{amsmath}
\usepackage{amssymb}
\usepackage{geometry}
\geometry{verbose,tmargin=1cm,bmargin=1cm,lmargin=2cm,rmargin=2cm}
\usepackage{graphicx}
\usepackage[colorinlistoftodos]{todonotes}
\usepackage{amsthm}
\usepackage[linesnumbered,ruled,vlined]{algorithm2e} 

\newtheorem{lemma}{Лемма}
\newtheorem{statement}{Утверждение}
\newtheorem{remark}{Замечание}

\title{Intrinsic volumes algorithm with  strategic buyer in 2 dimensional case}

\begin{document}

\maketitle

\section{Обозначения}
\textbf{Введем следующие обозначения}:
\begin{enumerate}
    \item $\hat{\theta}$ - искомое значение параметра,
    \item $S_t$ - область возможных значений искомого параметра $\hat{\theta}$ на $t$-м шаге алгоритма,
    \item $S^{+}_t$ - область, получаемая алгоритмом в случае покупки,
    \item $S^{-}_t$ - область, получаемая алгоритмом в случае отказа,
    \item $u_t$ - вектор параметров товара,
    \item $p_t$ - цена за товар, предложенная на на $t$ - м шаге алгоритма,
    \item $\eta_t$ - сдвиг цены,
    \item $\underline{p}_t = min(\{\theta^T u_t: \theta \in S_t\})$,
    \item $\underline{p}_t = max(\{\theta^T u_t: \theta \in S_t\})$,
    \item $A^{'}_t, A_t, P_t$ - соответсвенно нормаированная ($S^{'} = 2\sqrt{\pi S})$ и обычная площади, периметр.
\end{enumerate}

Выберем в качестве концов корзин (последовательное разбиение прямой отрезками) $l_k = a^{-k}$, где в дальнейшем получим условия на пармаетр.
\\
\\
\textbf{Условия на параметр} $a$:
\begin{enumerate}
    \item $a > 1,$
    \item $a^2 \leq \frac{1 - \gamma}{\pi (2 \gamma^r)},$
    \item $a^2 \leq \frac{1 - \gamma}{8 \pi (2 \gamma^r)},$
    \item $a^2 \leq \frac{1 - \gamma}{2 \gamma^r}.$
\end{enumerate}
\textbf{Выбор параметра сдвига} $\eta_t$:
\\
Выберем $\eta_t = \frac{\gamma^r}{1 - \gamma} diam(S_t).$
\\
\textbf{Условия, при которых работает алгоритм}:
\begin{enumerate}
    \item $1 < \frac{1 - \gamma}{\pi (2 \gamma^r)},$
    \item $1 < \frac{1 - \gamma}{8 \pi (2 \gamma^r)},$
    \item $1 < \frac{1 - \gamma}{2 \gamma^r}.$
\end{enumerate}
\section{Вспомогательные утверждения}
\lemma
Пусть в момент $t$ покупатель отказывается от приобретения, тогда в случае последующих $r$ пенализационных раундов будет выполнено следующее неравнство:
$$\hat{\theta}^T u_t \leq p_t + \frac{\gamma^r}{1 - \gamma} diam(S_t).$$
\begin{proof}
Пусть $S(\sigma_t)$ - прибыль покупателя при стратегии $\sigma.$
Рассмотрим 2 стратегии покупателя. При стратегии $\sigma^{'}_t$ - покупатель соглашется на покупку на раунде $t,$ а при стратегии $\sigma^{opt}_t$ - действует оптимальным образом (т.е. на раунде $t$ следует отказ).
Тогда выполнено следующее неравенство: $S(\sigma^{'}_t) \leq S(\sigma^{opt}_t).$
Подставим в это неравенство оценки для прибылей:
\begin{enumerate}
    \item $S(\sigma^{'}_t) = \gamma^{t - 1}(\hat{\theta}^T u_t - p_t),$
    \item $S(\sigma^{opt}_t) \leq \sum\limits_{s = t + r}^T \gamma^{s - 1}(\hat{\theta}^T u_t - \underline{p}_t) \leq \sum\limits_{s = t + r}^T \gamma^{s - 1}(\overline{p}_t - \underline{p}_t) \leq \sum\limits_{s = t + r}^T \gamma^{s - 1} diam(S_t) \leq \frac{\gamma^{t + r - 1}}{1 - \gamma} diam(S_t).$
\end{enumerate}
Получим следующее неравенство:
$\hat{\theta}^T u_t - p_t \leq \frac{\gamma^r}{1 - \gamma} diam(S_t).$
\end{proof}

\lemma
Пусть в момент $t$ покупатель приобретает предложенный товар, тогда выполнено следующее неравнство:
$$\hat{\theta}^T u_t \geq p_t - \frac{\gamma}{1 - \gamma} diam(S^{+}_t).$$
\begin{proof}
Пусть $S(\sigma_t)$ - прибыль покупателя при стратегии $\sigma,$а стратегия $\sigma^{opt}_t$ - оптимальная стратегия покупателя (т.е. на раунде $t$ следует соглашение на покупку).
Тогда выполнено следующее неравенство: $0 \leq S(\sigma^{opt}_t).$
Подставим в это неравенство оценку для оптимальной прибыли:
\begin{multline*}
    S(\sigma^{opt}_t) \leq \gamma^{t - 1}(\hat{\theta}^T u_t - p_t) + \sum\limits_{s = t + 1}^T \gamma^{s - 1}(\hat{\theta}^T u_t - \underline{p}_t) \leq \\ \leq \gamma^{t - 1}(\hat{\theta}^T u_t - p_t) + \sum\limits_{s = t + 1}^T \gamma^{s - 1}(\hat{\theta}^T u_t - \underline{p}(S^{+}_t)_t)) \leq \sum\limits_{s = t}^T \gamma^{s - 1} diam(S^{+}_t) \leq \frac{\gamma^{t}}{1 - \gamma} diam(S^{+}_t).
\end{multline*}
Получим следующее неравенство:
$\hat{\theta}^T u_t \geq p_t - \frac{\gamma}{1 - \gamma} diam(S^{+}_t).$
\end{proof}

\section{Алгоритм}
\begin{algorithm}[H]
\caption{Алгоритм для 2-мерного случая}
\SetAlgoLined
$\omega = (\overline{p}_t - \underline{p}_t)$\;
$h_{max} = $ максимальная длина сегмента $S_t \cap \{x: (u_t, x) = p\}$\;
\If{$\omega < 1 / T$}{
    $p_t = \underline{p}_t$\;
}\ElseIf{$bkt(A^{'}_t) = bkt(P_t) = k$}{
    выбираем цену $p_t: Area(S^{-}_t(p = p_t + \eta_t + \frac{\gamma^r}{1 - \gamma}diam(S_t))) = l^2_{k + 1} / (4 \pi)$\;
}\ElseIf{$bkt(A^{'}_t) = r > bkt(P_t) = k, \omega < h_{max}$}{
    выбираем цену $p_t: Area(S^{-}_t(p = p_t + \eta_t + \frac{\gamma^r}{1 - \gamma}diam(S_t))) = l^2_{k + 1} / (4 \pi)$\;
}\ElseIf{$bkt(A^{'}_t) = r > bkt(P_t) = k, \omega \geq h_{max}$}{
    выбираем цену $p_t: Perim(S_t) - Perim(S^{+}_t(p = p_t + \eta_t + \frac{\gamma^r}{1 - \gamma}diam(S_t)) = l_{k + 1} / 2$\; 
}
предлагаем покупателю $p_t + \eta_t$\;
\If{в случае согласия покупателя}{
    $S_{t + 1} = S^{+}_{t}(p = p_t + \eta_t - \frac{\gamma}{1 - \gamma})$\;
}{
    $S_{t + 1} = S^{-}_{t}(p = p_t + \eta_t + \frac{\gamma^r}{1 - \gamma})$\;
}
\end{algorithm}

\section{Осуществимость алгоритма}
\subsection{Случай $bkt(A^{'}_t) = bkt(P_t) = k$}
\statement
В случае выполнения условия $bkt(A^{'}_t) = bkt(P_t) = k$ цена, указанная в алгоритме существует, т.е. выполнены условия:
\begin{enumerate}
    \item $Area(S^{-}_t[p = \underline{p}_t + \eta_t + \frac{\gamma^r}{1 - \gamma}diam(S_t)]) \leq l^2_{k + 1} / (4 \pi),$
    \item $\underline{p}_t + \eta_t + \frac{\gamma^r}{1 - \gamma}diam(S_t) \leq \overline{p}_t.$
\end{enumerate}
\begin{proof}
1. Выполнена следующая оценка: $diam(S_t) \leq P_t / 2 \leq l_k / 2.$ Тогда, погружая $S_t$  прямоугольник со сторонами $\frac{\gamma^r + \gamma}{1 - \gamma}diam(S_t)$ и $diam(S_t)$, получаем:
\begin{gather*}
Area(S^{-}_t[p = \underline{p}_t + \eta_t + \frac{\gamma^r}{1 - \gamma}diam(S_t) = \underline{p}_t + \frac{\gamma^r + \gamma}{1 - \gamma}diam(S_t)]) \leq \\ \leq  (\frac{\gamma^r + \gamma}{1 - \gamma}diam(S_t)) diam(S_t) \leq l^2_k / 4 \frac{\gamma^r + \gamma}{1 - \gamma}.  
\end{gather*}
Таким образом для выполнения неравенства 1. досточно выполнения:
$l^2_k / 4 \frac{\gamma^r + \gamma}{1 - \gamma} \leq l^2_{k + 1} / (4 \pi).$ Подставляя в качестве $l_k = c a^{-k}$, получим $\pi a^2 \leq \frac{1 - \gamma}{\gamma + \gamma^r}.$ C учетом условия $a > 1$ получим следующее неравенство $\pi \leq \frac{1 - \gamma}{\gamma + \gamma^r}.$
\\
2. Знаем, что 
$$diam(S_t)(\overline{p}_t - \underline{p}_t) \geq Area(S_t) \geq l^2_{k + 1} / (4 \pi).$$ 
Отсюда следует, что $\overline{p}_t - \underline{p}_t \geq \frac{l^2_{k + 1}}{4 \pi diam(S_t)}.$
Для выполнения неравенства 2. достаточно показать:
$$diam(S_t) \leq \frac{1 - \gamma}{\gamma + \gamma^r}(\overline{p}_t - \underline{p}_t).$$
При этом мы знаем, что $diam(S_t) \leq l_k / 2,$ а $\overline{p}_t - \underline{p}_t \geq \frac{l^2_{k + 1}}{4 \pi diam(S_t)}$.

Тогда для выполнения 2. достаточно выполнения:
$$l^2_k / 4 \leq \frac{1 - \gamma}{\gamma + \gamma^r}l^{2}_{k + 1} / (4 \pi),$$
а это выполнено в силу выбра параметра $a$.
\end{proof}

\remark
На самом деле 2. следует из 1., т.к. уже из первого пункта мы знаем, что при отсечении части $S_t$ высотой $h = \eta_t + \frac{\gamma^r}{1 - \gamma} diam(S_t)$ мы получим кусок площади $l^2_{k + 1}$. Если бы $h$ было бы равно $\overline{p}_t - \underline{p}_t,$ мы бы получили кусок площади большей площади, т.к. $bkt(A^{'}_t) = bkt(P_t) = k.$

\subsection{Случай $bkt(A^{'}_t) = r > bkt(P_t) = k, \omega < h_{max}$}
\statement
В случае выполнения условия $bkt(A^{'}_t) = r > bkt(P_t) = k, \omega < h_{max}$ цена, указанная в алгоритме существует, т.е. выполнены условия:
\begin{enumerate}
    \item $Area(S^{-}_t[p = \underline{p}_t + \eta_t + \frac{\gamma^r}{1 - \gamma}diam(S_t)]) \leq l^2_{r + 1} / (4 \pi),$
    \item $\underline{p}_t + \eta_t + \frac{\gamma^r}{1 - \gamma}diam(S_t) \leq \overline{p}_t.$
\end{enumerate}
\begin{proof}
1. Заметим, что выполнена следующая оценка $diam(S_t) \leq \sqrt(2) l_r,$ т.к. можем погрузить $S_t$ в квадрат со стороной $diam(S_t).$
Тогда $Area(S_t) \leq diam(S_t)^2.$
Выполнены следующие неравенства:
\begin{multline*}
    Area(S^{-}_t[p = \underline{p}_t + \eta_t + \frac{\gamma^r}{1 - \gamma}diam(S_t) = \underline{p}_t + \frac{\gamma^r + \gamma}{1 - \gamma}diam(S_t)]) \leq \frac{\gamma^r + \gamma}{1 - \gamma}diam(S_t)^2 \leq \frac{\gamma^r + \gamma}{1 - \gamma} 2 l^2_r.
\end{multline*}
Тогда для выполнения неравенства из пункта 1. достаточно выполнения неравенства 
\begin{gather*}
    \frac{\gamma^r + \gamma}{1 - \gamma} 2 l^2_r \leq \frac{l^2_{r + 1}}{4 \pi},\\
    a^2 \leq \frac{1 - \gamma}{(\gamma + \gamma^r)(8 \pi)}.
\end{gather*}
Вспоминая условие $a > 1$, получим следующее неравенство:
$$8 \pi < \frac{1 - \gamma}{\gamma^r + \gamma}.$$

2. Условие 2. выполенно по тем же соображениям, что и в пункте 1.
\end{proof}

\subsection{Случай $bkt(A^{'}_t) = r > bkt(P_t) = k, \omega \geq h_{max}$}
\statement
В случае выполнения условия $bkt(A^{'}_t) = r > bkt(P_t) = k, \omega \geq h_{max}$ цена, указанная в алгоритме существует, т.е. выполнены условия:
\begin{enumerate}
    \item $P(S^{-}_t[p = \underline{p}_t + \eta_t + \frac{\gamma^r}{1 - \gamma}diam(S_t)]) \leq l_{k + 1},$
    \item $\underline{p}_t + \eta_t + \frac{\gamma^r}{1 - \gamma}diam(S_t) \leq \overline{p}_t.$
\end{enumerate}
\begin{proof}
1. Заметим, что $P(S^{-}_t[p = \underline{p}_t + \eta_t + \frac{\gamma^r}{1 - \gamma}]) \leq 2 diam(S_t) \frac{\gamma^r}{1 - \gamma} + 2 h_{max},$ т.к. мы можем погрузить $S^{-}_t$ в прямоугольник со сторонами $(h_{max}, diam(S_t) \frac{\gamma^r}{1 - \gamma}).$

Также из неравенств $$h^2_{max} / 2 \leq (\overline{p}_t - \underline{p}_t) h_{max} / 2 \leq Area(S_t) \leq \frac{l^2_r}{4 \pi}$$ следует, что $h_{max} \leq l_r / (2 \pi) \leq l_{k + 1} / (2 \pi).$

Тогда $P(S^{-}_t[p = \underline{p}_t + \eta_t + \frac{\gamma^r}{1 - \gamma}]) \leq 2 l_k / 2 \frac{\gamma^r}{1 - \gamma} + l_{k + 1} / \sqrt{\pi},$ т.к. $diam(S_t) \leq l_k / 2.$ Подставляя в качестве $l_k = a^{-k},$ получаем, что для выполнения требуемого неравенства необходимо выполнение условия:
$$a \frac{\gamma^r + \gamma}{1 - \gamma} \leq 1.$$

C учетом неравенства $a > 1,$ получаем условие $1 < \frac{1 - \gamma}{\gamma^r + \gamma}.$

2. Пользуемся теми же соображениями.
\end{proof}

\section{Оценка Regret}
\statement
При условиях на параметр $a,$ указанных выше, при значении $r,$ соотвествующем корректной работе алгоитма, будет выполнено следующее неравенство $Regret = O(\log{T}).$
\begin{proof}
1. Рассмотрим раунд типа $\overline{p}_t - \underline{p}_t < 1 / T.$ Суммарная ошибка во всех таких раундах $Regret_1 \leq T (\overline{p}_t - \underline{p}_t) \leq \frac{T}{T} = O(1).$

2. Рассмотрим раунд типа $bkt(A^{'}_t) = bkt(P_t) = k,$ где $k$ - фиксировано. Количество таких раундов $\leq \log{T},$ т.к. $\overline{p}_t - \underline{p}_t \leq l_k = a^{-k} > 1 / T,$ иначе мы перейдем к случаю 1. Теперь оценим потери покупателя на каждом из таких раундов.

В случае отказа, в силу ввыбора цены в алгоритме, мы переходим к случаю $bkt(P_{t + 1}) = k + 1,$ а потери оцениваются $r O(1) = O(1).$ При этом в силу леммы будет выполнено $$\hat{\theta}^T u_t \leq p_t + \eta_t \frac{\gamma^r}{1 - \gamma} diam(S_t),$$ т.е. мы перешли в область содержащее значение $\hat{\theta}.$

В случае согласия покупателя ошибка оценивается величиной $\overline{p}_t - \underline{p}_t \leq l_k / 2.$ При этом в силу леммы будет выполнено $$\hat{\theta}^T u_t \geq p_t + \eta_t - \frac{\gamma}{1 - \gamma} diam(S^{+}_t),$$ т.е. мы перейдем в область $S_{t + 1},$ содержащее истинное знчение параметра $\hat{\theta}.$

Теперь оценим величину $Area(S_t) - Area(S_{t + 1}):$
\begin{gather*}
    Area(S_t) - Area(S_{t + 1}) = Area(S_t) - Area(S^{+}_t(p = p_t + \eta_t - \frac{\gamma}{1 - \gamma}diam(S_t))) \geq \\
    \geq Area(S^{-}_t(p = p_t + \eta_t + \frac{\gamma^r}{1 - \gamma}diam(S_t))) - \\
    - Area(S^{-}_t(p = p_t + \eta_t + \frac{\gamma^r}{1 - \gamma}diam(S_t)) \setminus S^{-}_t(p = p_t + \eta_t - \frac{\gamma}{1 - \gamma}diam(S_t))) \geq \\
    \geq Area(S^{-}_t(p = p_t + \eta_t + \frac{\gamma^r}{1 - \gamma}diam(S_t))) - \frac{\gamma^r + \gamma}{1 - \gamma} diam(S_t) h_{max} \geq \\
    \geq \frac{l^2_{k + 1}}{4 \pi} - \frac{\gamma + \gamma^r}{1 - \gamma} \frac{l^2_k}{4},
\end{gather*}
т.к.
$diam(S_t) \leq l_k / 2, h_{max} \leq diam(S_t),$ а $Area(S^{-}_t(p = p_t + \eta_t + \frac{\gamma^r}{1 - \gamma}diam(S_t)) \setminus S^{-}_t(p = p_t + \eta_t - \frac{\gamma}{1 - \gamma}diam(S_t))),$ мы оценили площадью прямоугольника со сторонами $(h_{max}, \frac{\gamma^r + \gamma}{1 - \gamma}diam(S_t)).$

Тогда количество шагов, необходимых для того, чтобы покинуть случай $bkt(A^{'}_t) = k$ оценивается сверху
\begin{gather*}
    \frac{l^2_k / (4 \pi) - l^2_{k + 1} / (4 \pi)}{l^2_{k + 1} - \frac{(\gamma^r + \gamma) l^2_k}{(1 - \gamma) 4}} \leq \frac{1}{4 \pi} \frac{1 - a^{-2}}{a^{-2} - \pi \frac{\gamma^r + \gamma}{1 - \gamma}} = O(1).
\end{gather*}
В таком случае потери при согласии покупателя в случае $bkt(A^{'}_t) = k$  оцениваются сверху величиной $(\overline{p}_t - \underline{p}_t) O(1) \leq l_k / 2 O(1) = O(1).$

Таким образом суммарная потеря по всем раундам $bkt(A^{'}_t) = bkt(P_t) = k$ оценивается $Regret_2 \leq \log{T} O(1) = O(\log{T}).$

3. Рассмотрим раунд типа $bkt(A^{'}_t) = bkt(P_t) = k,$
\end{proof}
\end{document}

